%!TEX root = ./{{cookiecutter.project_name}}.tex

%% The main source file, place the text here.
%% If you need split the document into files
%% then includes it here using `input` command.
%%      For example:
%%      	\input{section01}
%% ~~~~~~~~~~~~~~~~~~~~~~~~~~~~~~~~~~~~~~~~~~~~~~~~~~~~~~~~~~~~~~~~~~~
%% Writing advice:
%%       Write every sentence in a separate line!
%%       It is simplifying the control versioning.
%%%%%%%%%%%%%%%%%%%%%%%%%%%%%%%%%%%%%%%%%%%%%%%%%%%%%%%%%%%%%%%%%%%%%%



\caption{
     СПЕЦСЕМИНАРЫ\\
     кафедры информационной безопасности\\
     весенний семестр 2019--2020 учебного года
}


\begin{center}
    \tabulinesep=2mm
    \begin{longtabu} to \textwidth {|c|X[0.5,l]|X[1,l]|X[2,l]|X[0.35, c]|X[0.1, c]|X[0.3,c]|}
        \hline % start
        \rowfont[c]{\bfseries}
        \textnumero&Руководители&Название&Аннотация&Дата и время&Ауд.&Первое заседание\\
        \hline
        \Index&М.~А.~Черепнёв,\linebreak И.~В.~Чижов& Математические проблемы криптографии с открытым ключом\linebreak
        (The mathematical problems of public key cryptography)& На спецсеминаре обсуждаются проблемы синтеза и анализа криптосистем с открытым ключом. Рассчитан на студентов 2-6 курса, аспирантов и сотрудников.\linebreak
        \textbf{Будет интересен студентам 1-2 курса. Студенты 2-го курса могут познакомиться с преподавателями кафедры ИБ.}& понедельник, 15:00&612&19.02.2019\\
        \hline% end
    \end{longtabu}
\end{center}

